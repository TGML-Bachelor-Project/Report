%\subsection{Model Implementation}
%\label{sec:Method:ModelImplementation}
%The proposed model is implemented using \hyperlink{https://pytorch.org/docs/1.9.1/}{Pytorch-1.9.1}\cite{PyTorchDocumentation} and contains the learnable parameters, $\textbf{z0} \in \mathbb{R}^{N \times 2}$, $\textbf{v0} \in \mathbb{R}^{N \times 2}$ and $\beta$. Here $\beta$ is a scalar representing the common bias term, $\textbf{z0}$ and $\textbf{v0}$ are tensors where $N$ is the number of unique nodes in the TGDN, $D$ is the size of the latent Euclidean space, which is chosen to be 2 for a 2-D latent space visualization of the TDGN, and $S$ is the number of steps in the model e.g. if the model tries to learn a representation, where the velocity can change 4 times, then S = 4.



% \subsection{Learning}
% \label{sec:Method:Learning}
% This section briefly describes the learning setup for the proposed model.
% The model learning is implemented using the \hyperlink{https://pytorch.org/ignite/}{Pytorch-Ignite-0.4.7}\cite{IgniteDocumentation} framework.
%  %The model training uses batching to handle the large amounts of data for TDGNs with many interactions. where each batch fed to the model contains $E$ rows for $E$ node interactions and 3 columns . which are fitted using backpropagation, based on the negative of the loglikelihood presented in equation  \ref{eq:LogLikelihoodFuncExplicit} in section \ref{sec:Method:LikelihoodFunc}









