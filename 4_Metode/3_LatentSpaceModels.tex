\subsection{Latent Space Models}
\label{sec:Method:LSM}
This subsection delves with explaining and understanding the latent space modelling approach.
It will (try to) describe what the latent space is, how specifically the euclidean latent space works and how the squared euclidean distance can and will be used as a metric for the velocity model.


\subsubsection{Latent Space}
\label{sec:Method:LSM:LatentSpace}
Latent space is an interesting concept, as it is intuitively hard to grasp.
Latent space is nothing, it doesn't really exist, and it is merely a construct on which information can be represented with different properties than they have originally.
\\
% Fra WIKI: A latent space, also known as a latent feature space or embedding space, is an embedding of a set of items within a manifold in which items which resemble each other more closely are positioned closer to one another in the latent space. Position within the latent space can be viewed as being defined by a set of latent variables that emerge from the resemblances from the objects. In most cases, the dimensionality of the latent space is chosen to be lower than the dimensionality of the feature space from which the data points are drawn, making the construction of a latent space an example of dimensionality reduction, which can also be viewed as a form of data compression or machine learning. 



\subsubsection{Euclidean Latent Space}
\label{sec:Method:LSM:EuclideanLatentSpace}
The Euclidean latent space is latent space governed by Euclidean geometry.
This means any measure of position, distance, velocity etc. are understood as Euclidean and can be computed as such. 
\\
Taking an offset in the nodes of a TDGN, and the modelling approach of this project, the properties of Euclidean space will be explained below.
\\\\
Given a TDGN consisting of N nodes, these are all placed in the Euclidean latent space. 
Positions in the Euclidean latent space are denoted as two-dimensional coordinates, and hence each node in the given TDGN is can be assigned a position, expressed for node $u$ as:
$\textbf{z}_u = \begin{pmatrix}
x_u\\
y_u
\end{pmatrix}$.
\\\\
As the project deals with temporally dynamic graph networks, the positions of nodes will naturally be temporally dependant, i.e. change over time. 
In order to accommodate this, each node is assigned a velocity vector, which entail they move at a constant velocity over a given time period.
For node $u$, the velocity vector is expressed as:
$\textbf{v}_u = \begin{pmatrix}
v_{x,u}\\
v_{y,u}
\end{pmatrix}$
By having a starting position, $z_u$, as well as a constant velocity, $v_u$, the position of a node at any time, $t$, is given by: 

\begin{equation}
    \textbf{z}_u(t) = \begin{pmatrix}
    x_u\\
    y_u
    \end{pmatrix}
    +
    \begin{pmatrix}
    v_{x,u}\\
    v_{y,u}
    \end{pmatrix}
    t
    = 
    \begin{pmatrix}
    x_u + v_{x,u}t\\
    y_u + v_{y,u}t
    \end{pmatrix}
    =
    z_u
\end{equation}

In the Euclidean latent space, based on the given positions, it is possible to compute distances using basic Pythagorean mathematics. 
In order to find the distance between two nodes, the most straightforward approach is to compute their reciprocal Euclidean distance.
The Euclidean distance, disregarding time, between node $u$ and $v$, denoted as $||z_u - z_v||_2$, can be computed from the following expression:

\begin{equation}
    ||\textbf{z}_u - \textbf{z}_v||_2
= 
\sqrt{(x_u - x_v)^2 + (y_u - y_v)^2}
\end{equation}

As seen above, the positions of nodes are time dependant, and hence this carries over to the distance measure, which is then expressed as:

\begin{equation}
    ||\textbf{z}_u(t) - \textbf{z}_v(t)||_2
= 
\sqrt{((x_u + v_{x,u}t) - (x_v + v_{x,v}t))^2 + ((y_u + v_{y,u}t) - (y_v + v_{y,v}t))^2}
\end{equation}

For mathematical reasons, which are explained lightly below, and in more detail under section \ref{sec:Method:IntensityFunc:LikelihoodFunc} regarding the likelihood function, this project utilizes the Squared Euclidean distance.

\subsubsection{Squared Euclidean Distance}
\label{sec:Method:LSM:SquaredEuclideanDistance}
%The Euclidean latent space, as described above, entailing spatial information about something that is not inherently spatial in nature, enables the computation of distance between the nodes.
As mentioned above, the N nodes of a TDGN are placed in the Euclidean latent space, and the properties they have as nodes in a network are approximately represented by the Euclidean measures of positions and velocities.
Their reciprocal distances governs the intensity of interaction between a given pair of nodes.
For this project, this reciprocal distance is calculated as the Squared Euclidean distance, which is expressed below, very similar to the standard Euclidean distance:

\begin{align} 
||\textbf{z}_u(t) - \textbf{z}_v(t)||_2^2
&= 
\left(\sqrt{((x_u + v_{x,u}t) - (x_v + v_{x,v}t))^2 + ((y_u + v_{y,u}t) - (y_v + v_{y,v}t))^2}\right)^2
\\
&=
((x_u + v_{x,u}t) - (x_v + v_{x,v}t))^2 + ((y_u + v_{y,u}t) - (y_v + v_{y,v}t))^2
\\
&=
(x_u - x_v + (v_{x,u} - v_{x,v})t)^2 + (y_u - y_v + ( v_{y,u} - v_{y,v})t)^2
\label{eq:SquaredEuclideanDistance}
\end{align}


