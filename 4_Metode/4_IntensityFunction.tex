\subsection{Intensity Function}
\label{sec:Method:IntensityFunc}
For the SCVM, an intensity function is defined for each pair of nodes in a given dynamic network.
The intensity function $\lambda_{u,v}$, governing the intensity of interaction between nodes $u$ and $v$, is written as the following:

\begin{equation}
    \lambda_{u,v}(t)
    =
    \exp \left(\beta - ||\textbf{z}_u(t) - \textbf{z}_v(t)||_2^2\right)
    \label{eq:IntensityFunc}
\end{equation}
This function is governed by two important terms.
\\
The first, the distance term, is the squared Euclidean distance of the node pair in 2D Euclidean space, which is described earlier in section \ref{sec:Method:LSM:EuclideanLatentSpace}.
This term is vital in reflecting the intensity of interaction between nodes as their reciprocal distance.
As the positions of nodes, and hence their reciprocal distances, are dependent on both starting positions and velocities, these are parameters the model will learn based ultimately on this intensity function.
\\
The second term is the common bias term $\beta$, a scalar which too is a learnable model parameter. 
The bias term serves as the background intensity and is hence used in modelling the intensity of interaction between nodes regardless of their reciprocal distances in Euclidean space. 
In this way, a large bias term will mean that nodes have higher intensity of interaction regardless of their latent positions and the opposite is true for a small bias term.


\subsubsection{Integral of the Intensity Function}
\label{sec:Method:IntensityFunc:IntegralIntensityFunc}
The intensity function for a given pair of two nodes, $u$ and $v$, using the bias term $\beta$ and squared Euclidean distance as distance term is, as stated above, given by equation (\ref{eq:IntensityFunc}).
As seen earlier with equation (\ref{eq:SquaredEuclideanDistance}), the distance term for constant velocity dynamics can be written as a function of the starting position plus the velocity over time, and as such the node pair intensity function is given by:

\begin{equation}
    \lambda_{u,v}(t)
    =
    \exp \left(\beta - \left((x_u - x_v + (v_{x,u} - v_{x,v})t)^2 + (y_u - y_v + ( v_{y,u} - v_{y,v})t)^2\right)\right)
\end{equation}
In order to compute the log-likelihood, which is explained in the next section \ref{sec:Method:LikelihoodFunc}, the solution for the integral of the intensity function needs to be found. 
\\
The reason the squared Euclidean distance is utilized, as opposed to using the standard Euclidean distance, is that it enables for an exact, analytical integration of the stated intensity function.
Having an analytical solution for the integral is important for this project, as it enables the computation to be exact and run fast, hence improving scalability.
This integral is written below:

\begin{equation}
    \int_{t_0}^{t_{n}} \lambda_{u,v}(s) \mathrm{d}s 
    =
    \int_{t_0}^{t_n} \exp \left(\beta - \left((x_u - x_v + (v_{x,u} - v_{x,v})s)^2 + (y_u - y_v + ( v_{y,u} - v_{y,v})s)^2\right)\right) \mathrm{d}s
\end{equation}
For ease of interpretation, substitutions are made for the following (without them, the final solution is incredibly long):

\begin{align}
    x_u - x_v &= a
    \\
    y_u - y_v &= b
    \\
    v_{x,u} - v_{x,v} &= m
    \\
    v_{y,u} - v_{y,v} &= n
\end{align}
This yields a substituted integral looking as such:

\begin{equation}
    \int_{t_0}^{t_n} \lambda_{u,v}(s) \mathrm{d}s 
    =
    \int_{t_0}^{t_n} \exp \left(\beta - \left((a + m \cdot s)^2 + (b + n \cdot s)^2\right)\right) \mathrm{d}s
\end{equation}
What is evident here, is that this integral must have an analytical solution, meaning conventional approximation approaches will not be needed in order to evaluate it's value. 
This is specifically due to the fact that the integration is happening over the exponential function of a function of quadratic form.
The analytical solution to this integral can be computed using a tool such as Maple and yields the following:

\begin{align}
    \int_{t_0}^T \lambda_{u,v}(s) \mathrm{d}s
    = 
    -\frac{\sqrt{\pi}}{2 \sqrt{m^{2}+n^{2}}}
    \cdot
    \exp\left(\frac{\left(-b^{2}+\beta\right) m^{2}+2abmn-n^{2}(a^{2}-\beta)}{m^{2}+n^{2}}\right)
    \\
    \cdot 
    \left(
    \operatorname{erf}\left(\frac{\left(m^{2}+n^{2}\right)t_{0}+am+b n}{\sqrt{m^{2}+n^{2}}}\right)
    -\operatorname{erf}\left(\frac{\left(m^{2}+n^{2}\right)T+am+b n}{\sqrt{m^{2}+n^{2}}}\right)
    \right)
    \label{eq:analytical_integral}
\end{align}
As can be seen above, the solution contains the Gauss error function denoted by $erf$.
This error function is a complex function of a complex variable, defined as:

\begin{equation}
\operatorname{erf} z=\frac{2}{\sqrt{\pi}} \int_{0}^{z} e^{-t^{2}} \mathrm{d} t
\end{equation}
What is important about the error function is that while it is in fact \textit{not} of a closed form, and hence disrupts the analytical nature of the integral, very efficient approximations can be made that eliminate the inconvenience of its non-closed form \cite{Ren2007Closed-formScience}.
