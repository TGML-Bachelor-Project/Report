\documentclass{article}
\usepackage[utf8]{inputenc}
\usepackage{graphicx}
\usepackage{amsmath}


%%%%%%%%%%Positioning of graphics figures
\usepackage{float}
\usepackage{subcaption}
\usepackage{amssymb,amsthm,amsmath}   
\graphicspath{{0_images/}}
\usepackage{geometry}
\usepackage[breaklinks]{hyperref}
\usepackage[table,xcdraw]{xcolor}


%Adding bibliography to table of contents
\usepackage[nottoc]{tocbibind}
%Adding bibtex reference file
\usepackage[backend=biber,sorting=ynt]{biblatex}
\addbibresource{references.bib}


%Custom file where we put terms and definitions
\usepackage{terms}
\setlength\parindent{0pt}

%Custom command for commenting out multiple lines
\newcommand{\comment}[1]{}

%Abbreviations package
%\usepackage[acronym]{glossaries}

%Center figure and table captions
\usepackage{caption}

%Formatting code
\usepackage{listings}
\lstset{
    inputencoding=utf8,
    extendedchars=true,
    literate=   {"}{{\\o}}1
                {ä}{{\"a}}1
                {ü}{{\"u}}1
}





\begin{document}

\begin{titlepage}
\center % Center everything 
\textsc{\LARGE Danmarks Tekniske Universitet}
\\[1cm]
\textsc{\huge{Bachelor Project}}
\\[1cm]
\begin{figure}[H]
    \centering
    \includegraphics[width=10em]{dtu_logo.png}
\end{figure}
\\[2cm]
\textsc{\huge{Scalable Machine Learning}}
\\[0.5cm] 
\textsc{\huge{for}}
\\[0.5cm] 
\textsc{\huge{Temporal Dynamic Graph Networks}}
\\[1.0cm]
\textsc{August Semrau Andersen - s183918
\\ William Diedrichsen Marstrand - s183921} 
\\[10pt]
%\textsc{\LARGE } \\ \textsc{\LARGE} \\[0.5cm]
%\includegraphics[width=9.5cm]{images/}

\medskip
%Number of characters: 12600, Standard Pages: 5.25
\null
\vfill
\vspace{0.5cm}
\begin{center}
\today,
 Danmarks Tekniske Universitet\\
\end{center}
\end{titlepage}


\newpage
\pagenumbering{gobble}


\subsection*{Abstract}

This project proposes the Stepwise Constant Velocity Model (SCVM) which models dynamic networks in two-dimensional latent space using Newtonian dynamics with stepwise computation.
The SCVM was implemented with a vectorized training setup in order to utilize the power of CUDA that stems from parallelization.

The proposed model was evaluated in terms of modelling capabilities, and was found to model dynamic networks well, using Newtonian dynamics in a stepwise fashion as to accommodate for the 

Running times were evaluated for the SCVM running on CPU and CUDA, the results concluding that vectorized training setup greatly improves running times when utilizing CUDA.
This scalable implementation enabled for the modelling of larger dynamic networks than would have been feasible with a non-parallelized training setup.

The learned Newtonian dynamics of a given dynamic network were visualized via the creation of animation which depicted the stepwise constant velocity movements of nodes in latent space.
These animations were improved through the implementation of non-disruptive corrections to the learned dynamics, elevating the interpretability and explainability of the modelled dynamic network.


The paper discusses these results and argues for the investigation of number of aspects of the proposed SCVM which could improve its modelling capabilities, scalability and the explainability of resulting visualizations.

Lastly, possible use cases for the proposed model are discussed, amongst which 



\newpage
\pagenumbering{arabic}

\tableofcontents
\clearpage

\comment{
\makeglossaries
\newacronym{vc}{VC}{Voice Conversion}
\newacronym{stt}{STT}{Speech-To-Text}
\newglossaryentry{vcmodel}
{
        name=VC model,
        description={Is a model that converts one speakers voice to another speakers voice.}
}

\newglossaryentry{sttmodel}
{
        name=STT model,
        description={Is a model that transcribes spoken words into written words.}
}

\printglossary[type=\acronymtype]
\printglossary
}




%%%%%%%%%%%%%%%%%%%%%%%%%%%%%%%%%%%%%%%%%%%%%%%%%%%%%%%%%%%%%%%%%%%%%%%%%%%%
%%%%%%%%%%%%%%%%%%%
\section*{Acronyms}

\textbf{TDGN} - Temporally Dynamic Graph Network
\\\\
\textbf{PP} - Possion Process






%%%%%%%%%%%%%%%%%%%
\section*{Glossary}

\textbf{VC model} - Is a model that converts one speakers voice to another speakers voice.
\\\\
\textbf{STT model} - Is a model that transcribes spoken words into written words.

%\printglossary[type=\acronymtype]
%\printglossaries
%\section*{Glossary}
\clearpage

\section{Introduction}
\label{sec:Intro}

%This section introduces the project and the motivations behind it.
%It will lightly shed light on the technical background of the project, introducing machine learning on graph networks.
%A review of previous, related work is given, followed by stating and explaining the present research questions.
%Lastly, an outline of the contents of the project will be given.


\subsection{Motivation}
\label{sec:Intro:Motivation}
Many real world systems and problems can be represented as static graph networks, a representation that enables both a human and machine-interpretable understanding of the given system, further allowing for the use of various machine learning tasks such as link prediction 
%(Thomas N. Kipf and Max Welling. Variational Graph Auto-Encoders. 2016. arXiv: 1611.07308 [stat.ML], Aditya Grover and Jure Leskovec. Node2vec: Scalable Feature Learning for Networks. New York, NY, USA, 2016. doi: 10.1145/2939672.2939754. url: https://doi.org/10.1145/2939672.2939754., William L. Hamilton, Rex Ying, and Jure Leskovec. Representation Learning on Graphs: Methods and Applications. 2018. arXiv: 1709.05584 [cs.SI].)
.
Hence, in order to grasp the structure of a social network such as Facebook, a static graph network representation seems appropriate.
Yet, to fully grasp such a network system, the aspect of time is crucial.
In the case of Facebook, taking a static snapshot, ie. a static graph representation, will show us how it's users relate to each other for one instant, but next to nothing about how these users interact.
It is evident that to properly understand interactions between users, say how they make new connections and break old, the the graph network representation must be temporally dynamic.

This project is motivated by this fact, and works solely with networks that change over time, which can be represented as temporally dynamic graph networks (TDGNs).
Approaches to modelling TDGN's have in recent years proved successful in allowing for well performing machine learning task
%(Srijan Kumar, Xikun Zhang, and Jure Leskovec. Predicting Dynamic Em-bedding Trajectory in Temporal Interaction Networks. New York, NY,USA, 2019. doi: 10.1145/3292500.3330895. url: https://doi.org/10.1145/3292500.3330895., Rakshit Trivedi et al. DyRep: Learning Representations over Dynamic Graphs. 2019. url: https://openreview.net/forum?id=HyePrhR5KX., da Xu et al. Inductive representation learning on temporal graphs. 2020. url: https://openreview.net/forum?id=rJeW1yHYwH., Emanuele Rossi et al. Temporal Graph Networks for Deep Learning onDynamic Graphs. 2020. arXiv: 2006.10637 [cs.LG].)
, yet most lack explainability and the ability for human-level understanding.
This project is further motivated by this, and seeks to investigate modelling of TDGNs in a manner that enforces explainability. 

A very recent work was developed by Simon Tommerup et. al. \cite{Tommerup2021LearningNetworks}, models TDGN's with while enforcing explainability, by utilizing a simpler modelling approach based on Newtonian dynamics in the latent space.
The work though suffers in terms of information loss, ie. the amount of information is kept during modelling, and severely in terms of scalability.
On the basis of this Newtonian dynamics modelling approach, the fundamental goal with this project is to expand it's modelling capabilities, which allow for explainability and visualization, while improving scalability, making them suitable for use on larger, more complex TDGN data.



\subsection{Dynamic Graph Network representations}
\label{sec:Intro:DynamicGraphNetworks}




\subsection{Related Work}
\label{sec:Intro:RelatedWork}
The work presented in this project is related to, and builds upon, many principles established in earlier work.

The first work to establish the use of Latent Space as a means of modelling graph networks was Hoff P et. al. \cite{Hoff2002LatentAnalysis}, in which the concept of depicting a static graph as nodes in latent space, with their relative distances describing likelihood of interaction, was introduced.

In "Dynamic Social Network Analysis using Latent Space Models" by Sarkar and Moore \cite{Sarkar2005DynamicModels}, the latent space approach was expanded to dynamic graph networks.
%, and the Discrete Diffusion model introduced.
%The Discrete Diffusion model entails 
The approach developed by Sarkar and Moore serves as an early foundation for this project in terms of modelling TDGNs in an explainable manner.

Several publications have since Sarkar and Moore been made, all presenting latent modelling approaches for dynamic graph networks. 
Worth mentioning are DyRep, Rakshit Trivedi 2019 et. al. \cite{TrivediDYREP:GRAPHS}, which like this project uses a temporal point process loss, and TGN, Emanuele Rossi 2020 et. al. \cite{RossiTEMPORALGRAPHS}, which .
 
One of the most recent works, on which this project builds, was developed by Simon Tommerup et. al. \cite{Tommerup2021LearningNetworks}.
His work seeks to simplify the latent space approach, by utilizing the simple Poisson point process, and enforces explainability by utilizing Newtonian dynamics in the latent space.




\subsubsection{Research Questions} 
\label{sec:Intro:ResearchQs}
The main scope, and essentially the overarching research question of this project lies in determining how a scalable and explainable model can be implemented for visualizing interactions in a temporal dynamic graph network.
In order to answer this question, three research questions have been put forth.
The first of these is
\\\\
\hspace*{5mm} 1. How well can dynamic networks be modelled using a SCVM representation in latent Euclidean space, with stepwise event computation?
%\hspace*{5mm} 1. How well can TDGNs be modelled using a velocity-dependant representation in latent Euclidean space, with stepwise event computation, and how well can it model the evolution of the TDGNs?
\\
This research question will be explored in two steps:
First, the constant velocity model will be implemented, based in the state of the art approach by Simon Tommerup et. al., as it serves as the foundation of modelling TDGN's through learning a starting position and a constant velocity vector for each node.
Then the approach will be expanded with stepwise computation, meaning each node is attributed several velocity vectors in order to better reflect changes over time.
\\
Evaluating the modelling ability of these approaches involves testing how well the model is able to infer correct node interactions from unseen data, and it's ability to map data of entire node pairs when these are unseen during testing.
\\\\
\hspace*{5mm} 2. To what extend can the model be implemented in a scalable manner?
\\
The current approaches to modelling TDGN's are heavily limited in terms of computational cost and are unable to model larger networks on accessible hardware in a timely fashion.
A scalable model will be able to model TDGN's in a way that is much more computationally efficient than prior approaches, and allow for the modelling of larger, more complex networks.
\\
This research question will be answered by implementing the computations of the proposed model in a parallelizable manner, and evaluated based on the improvement in efficiency over conventional, non-parralized computation approach.
\\\\
\hspace*{5mm} 3. How can the model be visualized to enforce explainability?
\\
A key advantage to the modelling approach using latent space representations is it's explainability, ie. that it is interpretable by a human observer.
Expanding to a stepwise constant velocity modelling approach should allow for a more detailed interpretation of the complex behavior that a TDGN might have over it's temporal duration.
\\
In order to answer this research question, visualizations of the TDGN as learned by the modelling approach will be evaluated based on interpretability. 



\subsection{Project Outline}
\label{sec:Intro:ThesisOutline}
The project consist of 4 more sections.

Section 2, the methodology section, provides a thorough technical explanation of the work presented in this project.
The first half of this section explains the theoretical aspects underlying the proposed model.
This part also describes the way synthetic data is generated for testing the model.


The second half fleshes out the proposed model, giving a detailed description of the optimization taking place within it, and hence the learning of the stepwise, dynamical representation that the model produces.
An explanation of the scalability aspects is further given, as this too defines how the model functions, in a parallizable manner.
The sections ends with an outline of details relating to reproducibility of the present work, enabling for future work.

Section 3 presents and analyzes the results which relate to answering the three research questions.

Section 4 discusses the present work, and gives some context as to how the work may be utilized in real-life use-cases.

Section 5 is the conclusion, summarizing the key findings presented in the project, hence wrapping up.
\\\\
All code for the project is publicly available at:
\\
\href{https://github.com/TGML-Bachelor-Project/TGML}{\textcolor{blue}{https://github.com/TGML-Bachelor-Project/TGML}}






\clearpage



\section{Method}
\label{sec:Method}
This section regards the technical methodology used in this project.
\\
First, it will give an explanation of the graph networks in the context that this project utilizes. 
\\
It will then explain the central idea behind the velocity dependant model about which the project revolves. 
\\
From there, this section will dive deep into the different aspects that this model relies upon.
This both includes mathematical and statistical theory, as well as the methodology related to implementing the model in code.

\subsection{Graphs and Networks}
\label{sec:Method:Graphs}
The most common way of describing a network of anything, that could be of accounts on Facebook, the electrical grid, distribution of goods from warehouses to stores etc. is to use a graph.
A graph is a representation that consists of a set of $N$ nodes, the entities whose interactions we are trying to depict, and edges $E$ which represent said interactions.

\subsubsection{Static Graphs}
\label{sec:Method:Graphs:StaticGraphs}
The most common type of graph is the static graph with a representation of nodes and edges that never changes. 
In this way, the static graph can be seen as a snapshot of a given network at a single point in time.


\subsubsection{Dynamic Graphs}
\label{sec:Method:Graphs:DynamicGraphs}
Not all networks can be represented by static graphs though, and sometimes having graphs be non-static means that they make for a better representation of the network they are modelling. When talking about dynamic graphs this project refers to graphs that are subject to a sequence of \textit{updates}. Where an update is an action that \textit{inserts} or \textit{deletes} edges or nodes in the graph or actions that \textit{alter attributes} of nodes or edges.
\\
The dynamic aspect differ between graphs, as they can be changing in different dimensions. This project focuses entirely on the very commonly used graphs that are dynamic in the temporal dimension, \textit{temporal dynamics graphs}, i.e. they change over time, this is not always the case. Another example could be if we had a network representing the airports which a plane could directly fly to. Now the graph would be dynamic in the geographical position of the plane as the graph could change to show which airport the plane is connected to based on its current position. Say the graph for the plane being at the Reykjavik Airport on Iceland will then change when the plane gets to Los Angles.

\subsubsection{Temporal Dynamics}
\label{sec:Method:Graphs:TemporallyDynamicGraphs}













\subsection{Constant Velocity Model}
\label{sec:Method:VModel}

This subsection will describe the idea of the Constant Velocity Model, on which the Piecewise Constant Velocity Model is based, and lightly explain the theoretical aspects enabling its workings.
The section will not dive into the technical aspects of how it works, as these will be delved with in the following sections.
Instead, it serves as the foundation for understanding how these fit into the overall modelling of TDGNs.
\\
The fully technical explanation of the model used in this project is presented in section \ref{sec:Method:PiecewiseConstantVModel}.


\subsubsection{Constant Velocity Modelling Approach}
\label{sec:Method:VModel:ModellingApproach}

The overarching and most fundamental idea behind this project is to model the nodes of a TDGN as being positioned in a two-dimensional space. 
In this space, a form of latent representation the reciprocal distance of any pair of nodes shall govern the intensity with which they interact.
The basic intuition is that, the closer any pair of nodes are, the more likely they are to interact.
\\
%Given a chronologically ordered sequence of pairwise interactions with timestamps as input data, the $N$ nodes of a given TDGN will be modelled in the Euclidean latent space.
For a TDGN, the intensity of interaction between nodes naturally changes over time, and so too does their positions and pairwise reciprocal distances.
The important understanding here is that given a pair of nodes interacting increasingly more, their reciprocal distance in two-dimensional latent space will decrease, and vice versa.
\\
This leads to the second-most fundamental idea behind this project, that the nodes' changing positions over time are represented as them each having a velocity. 
The intuition here is essentially that in the two-dimensional latent space, the velocity enables their positions to change over time, in order to reflect the change in intensity of interaction between nodes.
\\
These two ideas make up the fundamentals of the COnstant Velocity Model.
The intuition of how this works can though be hard to grasp, and hence an example is presented below.
If you got the idea already, feel free to skip the next paragraph.
\\\\
A short example:
\\
Say we have TDGN with two people on Facebook (the nodes), their interactions being sending each other messages (edges). 
The example story of these two is that one of them has dropped their wallet in the Metro, the other one picks it up and contacts the owner via Facebook, they exchange a bunch of messages relating to handing the wallet back, the wallet is handed back, and then they never write each other again.
In short terms, they have no interactions, they suddenly interact a lot, then they return to having no interactions.
\\
The interactions they make is now inputted into the velocity model as a chronologically ordered sequence of interactions, and they are placed as two nodes in the Euclidean latent space.
Here, the model will fit it's parameters to the input data, and when fitted assign each node a starting position, $Z$, and a starting velocity, $V$. 
The starting positions of the two nodes will be far apart, yielding very low intensity of interaction, as they at first do not interact whatsoever.
The example story tells us that they will at some point interact a lot for a short while, and so the model will assign them each a velocity that will make them intercept at a given time of high interaction intensity. 
In this scenario, the nodes will start out far apart, travel gradually closer to each around the halfway of the timespan, and then gradually regain reciprocal distance.
The distance in two-dimensional space will hence yield low, then high, then low intensity of interaction.
\\\\
FIGURE SHOWING TWO NODES MOVING AS EXPLAINED IN THE EXAMPLE ABOVE
\\\\
Using the Constant Velocity Model approach, the nodes of a TDGN can be represented as particles in two-dimensional latent space having starting positions and velocity. 
What this modelling approach provides, is essentially a physical representation of temporally dynamic graph network interactions.
\\\\
One last important understanding of the modelling approach, while perhaps clear from the above, is that the goal of the model is to learn the positions and velocities of each node in a given TDGN, based on the timestamped interactions of the TDGN, NOT the other way around. 
The modelling approach, and essentially the entire general idea behind this project, has the purpose of depicting a TDGN in observable space based on data of when pairs of nodes interact.
\\\\
The next four sections will dive into great detail about the texhnical aspects that enable the moddeling approach.
\\
First, section \ref{sec:Method:LSM:EuclideanLatentSpace} covers the actual two-dimensional latent space used in this project, namely Euclidean latent space. 
squared Euclidean distance is used as the distance measure for the intensity function, which governs the intensity of interaction, and so this will be covered in detail in section \ref{sec:Method:LSM:SquaredEuclideanDistance}.
\\
Second, a deep dive into Poisson statistics will be presented in section \ref{sec:Method:Poisson}.
The intensity of interaction between any pair of two nodes in a given TDGN is associated with it's own Poisson point process, and therefor essential to understand.
%Synthetic data generation also relies on Poisson statistics, and so this will also be explained fully.
\\
Thirdly, in section \ref{sec:Method:IntensityFunc}, the intensity function is explained in detail.
This function relies on the squared Euclidean distance, and introduces the Bias term, and is what converts between intensity of interaction and pairwise reciprocal distance of nodes in Euclidean latent space.
\\
Fourth, the likelihood function that is used for computing the log-likelihood on which the model optimizes and evaluates is explained in section \ref{sec:Method:LikelihoodFunc}.
\\
All of this is put together, and a thorough technical explanation of the Piecewise Constant Velocity Model is given.

\subsection{Latent Space Models}
\label{sec:Method:LSM}
This subsection delves with explaining and understanding the latent space modelling approach.
It will (try to) describe what the latent space is, how specifically the euclidean latent space works and how the squared euclidean distance can and will be used as a metric for the velocity model.


\subsubsection{Latent Space}
\label{sec:Method:LSM:LatentSpace}
Latent space is an interesting concept, as it is intuitively hard to grasp.
Latent space is nothing, it doesn't really exist, and it is merely a construct on which information can be represented with different properties than they have originally.
\\
% Fra WIKI: A latent space, also known as a latent feature space or embedding space, is an embedding of a set of items within a manifold in which items which resemble each other more closely are positioned closer to one another in the latent space. Position within the latent space can be viewed as being defined by a set of latent variables that emerge from the resemblances from the objects. In most cases, the dimensionality of the latent space is chosen to be lower than the dimensionality of the feature space from which the data points are drawn, making the construction of a latent space an example of dimensionality reduction, which can also be viewed as a form of data compression or machine learning. 



\subsubsection{Euclidean Latent Space}
\label{sec:Method:LSM:EuclideanLatentSpace}
The Euclidean latent space is latent space governed by Euclidean geometry.
This means any measure of position, distance, velocity etc. are understood as Euclidean and can be computed as such. 
\\
Taking an offset in the nodes of a TDGN, and the modelling approach of this project, the properties of Euclidean space will be explained below.
\\\\
Given a TDGN consisting of N nodes, these are all placed in the Euclidean latent space. 
Positions in the Euclidean latent space are denoted as two-dimensional coordinates, and hence each node in the given TDGN is can be assigned a position, expressed for node $u$ as:
$\textbf{z}_u = \begin{pmatrix}
x_u\\
y_u
\end{pmatrix}$.
\\\\
As the project deals with temporally dynamic graph networks, the positions of nodes will naturally be temporally dependant, i.e. change over time. 
In order to accommodate this, each node is assigned a velocity vector, which entail they move at a constant velocity over a given time period.
For node $u$, the velocity vector is expressed as:
$\textbf{v}_u = \begin{pmatrix}
v_{x,u}\\
v_{y,u}
\end{pmatrix}$
By having a starting position, $z_u$, as well as a constant velocity, $v_u$, the position of a node at any time, $t$, is given by: 

\begin{equation}
    \textbf{z}_u(t) = \begin{pmatrix}
    x_u\\
    y_u
    \end{pmatrix}
    +
    \begin{pmatrix}
    v_{x,u}\\
    v_{y,u}
    \end{pmatrix}
    t
    = 
    \begin{pmatrix}
    x_u + v_{x,u}t\\
    y_u + v_{y,u}t
    \end{pmatrix}
    =
    z_u
\end{equation}

In the Euclidean latent space, based on the given positions, it is possible to compute distances using basic Pythagorean mathematics. 
In order to find the distance between two nodes, the most straightforward approach is to compute their reciprocal Euclidean distance.
The Euclidean distance, disregarding time, between node $u$ and $v$, denoted as $||z_u - z_v||_2$, can be computed from the following expression:

\begin{equation}
    ||\textbf{z}_u - \textbf{z}_v||_2
= 
\sqrt{(x_u - x_v)^2 + (y_u - y_v)^2}
\end{equation}

As seen above, the positions of nodes are time dependant, and hence this carries over to the distance measure, which is then expressed as:

\begin{equation}
    ||\textbf{z}_u(t) - \textbf{z}_v(t)||_2
= 
\sqrt{((x_u + v_{x,u}t) - (x_v + v_{x,v}t))^2 + ((y_u + v_{y,u}t) - (y_v + v_{y,v}t))^2}
\end{equation}

For mathematical reasons, which are explained lightly below, and in more detail under section \ref{sec:Method:IntensityFunc:LikelihoodFunc} regarding the likelihood function, this project utilizes the Squared Euclidean distance.

\subsubsection{Squared Euclidean Distance}
\label{sec:Method:LSM:SquaredEuclideanDistance}
%The Euclidean latent space, as described above, entailing spatial information about something that is not inherently spatial in nature, enables the computation of distance between the nodes.
As mentioned above, the N nodes of a TDGN are placed in the Euclidean latent space, and the properties they have as nodes in a network are approximately represented by the Euclidean measures of positions and velocities.
Their reciprocal distances governs the intensity of interaction between a given pair of nodes.
For this project, this reciprocal distance is calculated as the Squared Euclidean distance, which is expressed below, very similar to the standard Euclidean distance:

\begin{align} 
||\textbf{z}_u(t) - \textbf{z}_v(t)||_2^2
&= 
\left(\sqrt{((x_u + v_{x,u}t) - (x_v + v_{x,v}t))^2 + ((y_u + v_{y,u}t) - (y_v + v_{y,v}t))^2}\right)^2
\\
&=
((x_u + v_{x,u}t) - (x_v + v_{x,v}t))^2 + ((y_u + v_{y,u}t) - (y_v + v_{y,v}t))^2
\\
&=
(x_u - x_v + (v_{x,u} - v_{x,v})t)^2 + (y_u - y_v + ( v_{y,u} - v_{y,v})t)^2
\label{eq:SquaredEuclideanDistance}
\end{align}



\input{4_Metode/4_PoissonProcess}
\subsection{Intensity Function}
\label{sec:Method:IntensityFunc}

For the constant velocity model, the intensity function is defined for each pair of nodes in the given TDGN.
The intensity function $\lambda_{u,v}$ related to a pair of two nodes of interest, $u$ and $v$, is written as the following:

\begin{equation}
    \lambda_{u,v}(t)
    =
    \exp \left(\beta - ||\textbf{z}_u(t) - \textbf{z}_v(t)||_2^2\right)
    \label{eq:IntensityFunc}
\end{equation}

This function is governed by two important terms.
\\
The first, the distance term, is the reciprocal Euclidean distance of the node pair, which is described earlier in section \ref{sec:Method:LSM:SquaredEuclideanDistance}.
This term is vital in reflecting the intensity of interaction between nodes as their reciprocal distance.
As the positions of nodes are dependent on both starting position and velocity, these are parameters the model will learn through optimizing based ultimately on this intensity function.
\\
The second is the bias term $\beta$, briefly mentioned in section \ref{sec:Method:VModel:IntensityFuncIntro}.
The bias term is a scalar which is a learnable model parameter which the model learns during training. The bias term serves as the background intensity, and hence models the intensity of interaction between nodes regardless of their reciprocal distance in Euclidean space. In this way a large bias term will mean that nodes have an increased likelihood of interacting regardless of their latent positions and the opposite is true for a small bias term.


% \subsubsection{Bias Term $\beta$}
% \label{sec:Method:IntensityFunc:BiasTerm}

% The bias term, $\beta$, a part of the intensity function, is a learnable parameter for the model.
% This project considers two different implementations of the bias term. One version uses a \textit{common bias term} with a single $\beta$ scalar parameter as explained for equation (\ref{eq:IntensityFunc}) above.
% An alternative to the common bias term is the \textit{node specific bias term} where a bias term is added for each node $i$, such that the scalar $\beta_i$ is a learnable model parameter for the bias of node $i$. This results in a model with a learnable parameter vector $\beta \in \mathbb{R}^N$ where $N$ is the number of nodes in the network. Then the intensity function is written as:
% \begin{equation}
%         \lambda_{u,v}(t)
%     =
%     \exp \left(\beta_u + \beta_v - ||\textbf{z}_u(t) - \textbf{z}_v(t)||_2^2\right)
%     \label{eq:NodeBiasIntensityFunc}
% \end{equation}

% For this report the common bias term model will be the primary implementation of consideration.

\subsubsection{Integral of the Intensity Function}
\label{sec:Method:IntensityFunc:IntegralIntensityFunc}

The intensity function for a given pair of two nodes, $u$ and $v$, using the bias term $\beta$ and squared Euclidean distance as distance term, is as stated above given by (\ref{eq:IntensityFunc}).
As seen earlier in (\ref{eq:SquaredEuclideanDistance}), the distance term can be written as a function of the starting position plus the velocity over time, as such:

\begin{equation}
    \lambda_{u,v}(t)
    =
    \exp \left(\beta - \left((x_u - x_v + (v_{x,u} - v_{x,v})t)^2 + (y_u - y_v + ( v_{y,u} - v_{y,v})t)^2\right)\right)
\end{equation}

In order to compute the log-likelihood function, which is explained in the next section \ref{sec:Method:LikelihoodFunc}, the solution for the integral of the intensity function needs to be found. 
\\
The reason the squared Euclidean distance is utilized, as opposed to using the standard Euclidean distance, is that it enables for an exact, analytical integration of the stated intensity function for constant velocity.
Having an analytical solution for the integral is important for this project, as it enables the computation to be exact and run fast.
\\
This integral is written below:

\begin{equation}
    \int_{t_0}^T \lambda_{u,v}(s) \mathrm{d}s 
    =
    \int_{t_0}^T \exp \left(\beta - \left((x_u - x_v + (v_{x,u} - v_{x,v})s)^2 + (y_u - y_v + ( v_{y,u} - v_{y,v})s)^2\right)\right) \mathrm{d}s
\end{equation}

For ease of interpretation, substitutions are made for the following (without them, the final solution is incredibly long):

\begin{align}
    x_u - x_v &= a
    \\
    y_u - y_v &= b
    \\
    v_{x,u} - v_{x,v} &= m
    \\
    v_{y,u} - v_{y,v} &= n
\end{align}

This yields a substituted integral looking as such:

\begin{equation}
    \int_{t_0}^T \lambda_{u,v}(s) \mathrm{d}s 
    =
    \int_{t_0}^T \exp \left(\beta - \left((a + m \cdot s)^2 + (b + n \cdot s)^2\right)\right) \mathrm{d}s
\end{equation}

What is evident here, is that this integral must have an analytical solution, meaning approximation will not be needed in order to evaluate it's value. 
This is specifically due to the fact that the integration is happening over the exponential function of a function of quadratic form.
\\\\
The analytical solution to this integral is computed to being:


\begin{align}
    \int_{t_0}^T \lambda_{u,v}(s) \mathrm{d}s
    = 
    -\frac{\sqrt{\pi}}{2 \sqrt{m^{2}+n^{2}}}
    \cdot
    \exp\left(\frac{\left(-b^{2}+\beta\right) m^{2}+2abmn-n^{2}(a^{2}-\beta)}{m^{2}+n^{2}}\right)
    \\
    \cdot 
    \left(
    \operatorname{erf}\left(\frac{\left(m^{2}+n^{2}\right)t_{0}+am+b n}{\sqrt{m^{2}+n^{2}}}\right)
    -\operatorname{erf}\left(\frac{\left(m^{2}+n^{2}\right)T+am+b n}{\sqrt{m^{2}+n^{2}}}\right)
    \right)
    \label{eq:analytical_integral}
\end{align}

As can be seen above, the solution contains the Gauss error function denoted by $erf$.
This error function is a complex function of a complex variable, and is defined as:

\begin{equation}
\operatorname{erf} z=\frac{2}{\sqrt{\pi}} \int_{0}^{z} e^{-t^{2}} \mathrm{d} t
\end{equation}

What is important about the error function, is that while it poses the solving of another set of integrals, these do not disrupt the analytical nature of the integral of the intensity function.

\input{4_Metode/6_LikelihoodFunction}

\subsection{Proposed Model}
\label{sec:Method:PiecewiseConstantVModel}

The model, which was explained in non-technical terms in section \ref{sec:Method:VModel}, will here be fleshed out in full technical detail.
This will combine the methodology presented in the previous 4 sections, including latent space models (\ref{sec:Method:LSM}), Poisson processes (\ref{sec:Method:Poisson}), the intensity function (\ref{sec:Method:IntensityFunc}) and the likelihood function (\ref{sec:Method:LikelihoodFunc}).


\subsubsection{Constant Velocity base Model}
\label{sec:Method:PiecewiseConstantVModel:ConstantVelocityModel}

The constant velocity model, which serves as the basis of this project's modelling approach, determines the position and velocity for N nodes in a TDGN, based on the pairwise interactions they make over a given timespan.
\\\\
The model takes as input a chronologically ordered sequence of $n$ timestamped, pairwise node interactions, which are stored as $n$ tuples $(u_i, v_i, t_i)$.
Here, $u_i$ and $v_i$ denote the two interacting nodes for interaction $i$, and $t_i$ denotes the relevant timestamp.
\\
The learnable parameters of the model are the set of parameter matrices $\textbf{Z}, \textbf{V} \in \mathbb{R} ^{N \times 2}$ as well as the bias term $\beta \in \mathbb{R}$.
The parameter matrix $\textbf{Z}$ contains starting positions for all $N$ nodes in the TDGN, holding the $x$-component in the first column, and the $y$-component in the second.
The same is true for $\textbf{V}$, just that it contains all velocities, which hence are constant over the given timespan.
The bias term $\beta$, as explained in section \ref{sec:Method:IntensityFunc:BiasTerm}, is a real number used in modelling the background intensity of interaction.
\\\\
Each pair of nodes, $u$ and $v$, are associated with a Poisson point process, described under section \ref{sec:Method:Poisson:PoissonPointProcess}.
These processes are each governed by an intensity function (\ref{eq:IntensityFunc}), described under section \ref{sec:Method:IntensityFunc}, which says that the closer $u$ and $v$ are, the higher the intensity of interaction will be. 
These pairwise intensity functions are utilized in the log-likelihood function \ref{eq:LogLikelihoodFuncExplicit}, described under section \ref{sec:Method:LikelihoodFunc}, which outputs the log-likelihood of the input data given the parameters of the model, matrices $\textbf{Z}$ and $\textbf{V}$, and $\beta$.
\\
The optimization problem of the model is to maximize the log-likelihood (in practice minimize the negated log-likelihood) of the input data by tweaking the model parameters.
In this regard, the goal is to learn the true initial conditions, $\textbf{Z}, \textbf{V}, \beta$, of the model.


\subsubsection{Expanding to Piecewise Constant Velocity Model}
\label{sec:Method:PiecewiseConstantVModel:PiecewiseConstantVelocityModel}

With the constant velocity model, it is theoretically possible to model interactions between $N$ nodes in a TDGN by finding their true starting positions and velocities, as well as the bias term $\beta$, for a given timespan.
\\\\
The positions of nodes are modelled to change, reflecting the change in intensity of interaction between nodes, starting in $\textbf{Z}$ at time $t=0$ and moving along the trajectories that result from the constant velocities $\textbf{V}$ over time, until end time $T$.
With only one set of parameters for the entire timespan, the modelling ability is limited to describing changes in intensity of interaction with only changes in positions along the the trajectory of the velocity vectors of $\textbf{V}$.
This will in many cases be insufficient for depicting the given TDGN in a detailed manner. 
In order to model with greater detail, the Constant Velocity Model is expanded to the Piecewise Constant Velocity Model. 
The central idea with this model is splitting the entire timespan of a TDGN into smaller pieces, and using several instances of the Constant Velocity Model to model each of these pieces, resulting in several sets of true parameters.
Having several sets of positions and velocities allows for the TDGN to be modelled in a much more detailed manner, and should produce better representations in the Euclidean latent space.
\\\\
As stated, the Piecewise Constant Velocity Model is in essence just a sequence of Constant Velocity Models, each modelling a piece of the entire timespan of which the given TDGN consists. 
This means that eaach node in the TDGN will be associated with several positions and velocities, meaning it changes direction of travel in the Euclidean latent space over the course of the entire TDGN.



\subsubsection{Regularizing Velocity Changes}




\subsection{Model Implementation}
\label{sec:Method:ModelImplementation}



\subsubsection{Sequencing singular Constant Velocity Models}
\label{sec:Method:ModelImplementation:}



\subsubsection{Learning}
\label{sec:Method:ModelImplementation:Learning}



\subsubsection{Scalability}
\label{sec:Method:ModelImplementation:Scalability}

\clearpage


\section{Results}

BASELINES: BETA OG Z0 såvel som simons arbejde!!! Average intensity of all dyads. 

\subsection{First Research Question}
\label{sec:ResearchQuestion1}
The first research question of this project states:
\\
"How  well  can  dynamic  networks  be  modelled  using  a  SCVM  representation  in latent Euclidean space, with stepwise event computation?"
\\\\
In order to answer this question, the modelling approach will be evaluated on synthetically generated data.
Here synthetic dataset 1 is utilized, which is synthesized using a single velocity vector, as explained in section \ref{sec:Data:SyntheticData:SyntheticDataset1}.

\subsubsection{Modelling of single-step synthetic data}
\label{sec:ResearchQuestion1:singleStepSynthetic}
The first part of answering research question one consists of confirming proper modelling performance of a single-step Constant Velocity Model(CVM). 
This is done as an initial procedure as it lays the foundation of the stepwise CVM, meaning if the single-step model does not work as intended, the stepwise will neither.
\\\\
\textbf{Loss and beta convergence results.}
The stepwise model has been trained for 50, 500 and 5000 epochs on the full dataset 1 as one training batch using a single velocity step and a learning rate of 0.025. The model is compared against a baseline model which is simply a model with no dynamics i.e. a model with no velocities. Also for comparison the ground truth is provided.
Below, table \ref{tab:SingleStep1}, shows the learned beta parameter and final negative log loss, including that of the ground truth model.

\begin{table}[H]
\centering
\begin{tabular}{|l|c|cc|}
\hline
Model         & \multicolumn{1}{l|}{Num. Epochs} & Beta & Avg. NLL \\ \hline
Ground Truth  & -                                & 7.5  & -82,000  \\
No V Baseline & 5000                             & XX   & XX       \\
1 Step Model  & 5                                & XX   & XX       \\
1 Step Model  & 5000                             & XX   & XX       \\ \hline
\end{tabular}
\caption{Final learned $\beta$ value and negative log loss for baseline trained model.}
\label{tab:SingleStep1}
\end{table}

\\\\
\textbf{Intensity rate comparison results.}
\\
The intensity rates for a 5000 epochs trained model on 

\begin{figure}[H]
    \centering
    \includegraphics[width=\textwidth]{0_images/synth1_epochs5000.png}
    \caption{Node pair interaction intensity for synthetic dataset 1 trained for 5000 epochs. Blue line is the ground truth model, red is the trained model.}
    \label{fig:RQ1synth1}
\end{figure}
\\\\
\textbf{Animation check.}
\\\\
\textbf{Interaction removal results.}
The results from interaction removal, as explained in \ref{sec:Method:Evaluation:AUC}, shows quite a bad 
\\\\
\textbf{Dyad removal results.}

The baseline that has no dynamics is killed here, as it is not even able to be intepreted for intensity rate comparison.




\subsubsection{Multi-step TDGN modelling of synthetic data}
\label{sec:ResearchQuestion1:multiStepSynthetic}
The second part of answering the first research question evaluates the SCVM with 10 steps to test its capabilities with multiple steps.
The test uses the synthetic dataset 2, see section \ref{sec:Data:SyntheticData:SyntheticDataset2}.
\\\\
\textbf{Loss and beta convergence results.}

\begin{table}[H]
\centering
\begin{tabular}{|l|cc|}
\hline
Num. Epochs   & Beta & NLL\\ \hline
Ground Truth & 7.5  & -6.29      \\
5          & XX   & XX       \\
50          & 7.438   & -6.208       \\
500          & XX   & XX       \\
2500          & XX   & XX       \\
5000          & XX   & XX       \\
\hline
\end{tabular}
\caption{Final learned $\beta$ value and negative log loss for trained model on synthetic dataset 2 \ref{sec:Method:Reproducibility:SyntheticDataset2}}
\label{tab:MultiStep1}
\end{table}

\noindent \textbf{Intensity rate comparison results.}
\\\\
\textbf{Animation check.}
\\\\
\textbf{Interaction removal results.}
The 
\\\\
\textbf{Dyad removal results.}



\subsubsection{Multi-step modelling of real world data}
\label{sec:ResearchQuestion1:ResistanceTraining}
The third part of answering the first research question investigates the SCVM's modelling capabilities on a real world dataset.
\\
Real dataset 1, see section \ref{sec:Data:RealData:RealDataset1}, is utilized here.
\\\\
\textbf{Animation check.}
\\\\
\textbf{Interaction removal results.}




\clearpage
\subsection{Second Research Question}
\label{sec:ResearchQuestion2}
The second research question of this project states:
\\
"To what extend can the model be implemented in a scalable manner?"



\subsubsection{Non-Vectorized vs. Vectorized model setup}
\label{sec:ResearchQuestion2:comparison}
In order to 



\subsubsection{150 node synthetic dataset}
\label{sec:ResearchQuestion2:150nodeSynthetic}

Synthetic dataset 3, see section \ref{sec:Method:Reproducibility:SyntheticDataset3}, is utilized here.
   
   

\subsubsection{986 node real dataset}
\label{sec:ResearchQuestion2:986nodeReal}

Real dataset 2, see section \ref{sec:Method:Reproducibility:RealDataset2}, is utilized here.
   
\clearpage


\subsection{Third Research Question}
\label{sec:ResearchQuestion3}

\clearpage

\section{Discussion}


\subsection{Discussion of Results}
\label{sec:DiscussionResults}

%%%%%%%%%%%%%%%%%%%%%%%%%%%%%%%%%%%%%%%%%%%%%%%%%%%%%%%%%%%%%%%%%%%%%%%%%%%%%%%%%%%%%%%%%%
\subsubsection{First Research Question}
\label{sec:DiscussionResults_Q1}



%%%%%%%%%%%%%%%%%%%%%%%%%%%%%%%%%%%%%%%%%%%%%%%%%%%%%%%%%%%%%%%%%%%%%%%%%%%%%%%%%%%%%%%%%%
\subsubsection{Second Research Question}
\label{sec:DiscussionResults_Q2}



%%%%%%%%%%%%%%%%%%%%%%%%%%%%%%%%%%%%%%%%%%%%%%%%%%%%%%%%%%%%%%%%%%%%%%%%%%%%%%%%%%%%%%%%%%
\subsubsection{Third Research Question}
\label{sec:DiscussionResults_Q3}

\clearpage






\section{Conclusion}
%Kun Opsummerende!
%Må ikke komme med nye resultater eller nye diskusioner.
%Lidt nyt må gerne være med i form af implikationer ved de resultater, som vi har fået og mulige fremtidige undersøgelser.

\clearpage

%\bibliographystyle{unsrt}
%\bibliography{references}
\printbibliography
\clearpage

\section*{Appendix}
\appendix
%Use appendix by inputting seperate files.
%DO NOT WRITE APPENDIX TEXT DIRECTLY HERE IT WILL BE TOO LONG

%test file to show how to add files to the appendix
%\input{appendix/sanity_check_speakers}



\end{document}
