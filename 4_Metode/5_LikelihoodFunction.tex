\subsection{Likelihood Function}
\label{sec:Method:LikelihoodFunc}
In order to learn a fitting representation of a given dynamic network, the proposed SCVM needs a metric on which it can train.
For this purpose, the likelihood function is introduced.
The likelihood function computes a metric on which the model optimization can be based, in this case relying heavily on the intensity function described above in section \ref{sec:Method:IntensityFunc}.
\\
This likelihood function computes the likelihood of the relevant input data, given a set of model parameters $\textbf{Z}$, $\textbf{V}$ and $\beta$.
In other words, what is the likelihood that the input data, the chronologically ordered pairwise node interactions, tuples of the form $(u, v, t)$, will occur by the SCVM with the  given parameters.
\\
For this project, the likelihood function, specifically is the \textit{log}-likelihood function $\ell$ given by the following expression:

\begin{equation}
    \ell = \sum_{i=1}^n \log \lambda (t_i) - \int_{t_0}^{t_n} \lambda(t) \mathrm{d} t
    \label{eq:LogLikelihoodFunc}
\end{equation}
This log-likelihood is the sum of the log of the intensity function for all interactions in the input data $\sum_{i=1}^n \log \lambda (t_i)$, minus the the integral of the intensity function over the entire timespan for all node pairs in the entire network, $\int_{t_0}^{t_n} \lambda(t) \mathrm{d} t$. 
In short, the log-likelihood is hence the log of the sum over event probabilities for all node pairs minus the log of the sum of the non-event probability for all node pairs, the probabilities being derived in section \ref{sec:Method:Poisson:EventProbability} and \ref{sec:Method:Poisson:NonEventProbability}.
A more elaborate way of writing the log-likelihood, while explicitly accounting for all node-pairs, is then:

\begin{equation}
    \ell = \sum_{i=1}^n \log \lambda_{u_i,v_i} (t_i) - \sum_{u=1}^{N-1} \sum_{v > u}^{N} \int_{t_0}^{t_n} \lambda_{u,v}(t) \mathrm{d} t
    \label{eq:LogLikelihoodFuncExplicit}
\end{equation}
Here the analytical solution from equation (\ref{eq:analytical_integral}) becomes essential for an efficient computation of the non-event probability sum.




