
\subsection{Constant Velocity Model}
\label{sec:Method:VModel}

This subsection will describe the idea of the Constant Velocity Model, on which the Piecewise Constant Velocity Model is based, and lightly explain the theoretical aspects enabling its workings.
The section will not dive into the technical aspects of how it works, as these will be delved with in the following sections.
Instead, it serves as the foundation for understanding how these fit into the overall modelling of TDGNs.
\\
The fully technical explanation of the model used in this project is presented in section \ref{sec:Method:PiecewiseConstantVModel}.


\subsubsection{Constant Velocity Modelling Approach}
\label{sec:Method:VModel:ModellingApproach}

The overarching and most fundamental idea behind this project is to model the nodes of a TDGN as being positioned in a two-dimensional space. 
In this space, a form of latent representation the reciprocal distance of any pair of nodes shall govern the intensity with which they interact.
The basic intuition is that, the closer any pair of nodes are, the more likely they are to interact.
\\
%Given a chronologically ordered sequence of pairwise interactions with timestamps as input data, the $N$ nodes of a given TDGN will be modelled in the Euclidean latent space.
For a TDGN, the intensity of interaction between nodes naturally changes over time, and so too does their positions and pairwise reciprocal distances.
The important understanding here is that given a pair of nodes interacting increasingly more, their reciprocal distance in two-dimensional latent space will decrease, and vice versa.
\\
This leads to the second-most fundamental idea behind this project, that the nodes' changing positions over time are represented as them each having a velocity. 
The intuition here is essentially that in the two-dimensional latent space, the velocity enables their positions to change over time, in order to reflect the change in intensity of interaction between nodes.
\\
These two ideas make up the fundamentals of the COnstant Velocity Model.
The intuition of how this works can though be hard to grasp, and hence an example is presented below.
If you got the idea already, feel free to skip the next paragraph.
\\\\
A short example:
\\
Say we have TDGN with two people on Facebook (the nodes), their interactions being sending each other messages (edges). 
The example story of these two is that one of them has dropped their wallet in the Metro, the other one picks it up and contacts the owner via Facebook, they exchange a bunch of messages relating to handing the wallet back, the wallet is handed back, and then they never write each other again.
In short terms, they have no interactions, they suddenly interact a lot, then they return to having no interactions.
\\
The interactions they make is now inputted into the velocity model as a chronologically ordered sequence of interactions, and they are placed as two nodes in the Euclidean latent space.
Here, the model will fit it's parameters to the input data, and when fitted assign each node a starting position, $Z$, and a starting velocity, $V$. 
The starting positions of the two nodes will be far apart, yielding very low intensity of interaction, as they at first do not interact whatsoever.
The example story tells us that they will at some point interact a lot for a short while, and so the model will assign them each a velocity that will make them intercept at a given time of high interaction intensity. 
In this scenario, the nodes will start out far apart, travel gradually closer to each around the halfway of the timespan, and then gradually regain reciprocal distance.
The distance in two-dimensional space will hence yield low, then high, then low intensity of interaction.
\\\\
FIGURE SHOWING TWO NODES MOVING AS EXPLAINED IN THE EXAMPLE ABOVE
\\\\
Using the Constant Velocity Model approach, the nodes of a TDGN can be represented as particles in two-dimensional latent space having starting positions and velocity. 
What this modelling approach provides, is essentially a physical representation of temporally dynamic graph network interactions.
\\\\
One last important understanding of the modelling approach, while perhaps clear from the above, is that the goal of the model is to learn the positions and velocities of each node in a given TDGN, based on the timestamped interactions of the TDGN, NOT the other way around. 
The modelling approach, and essentially the entire general idea behind this project, has the purpose of depicting a TDGN in observable space based on data of when pairs of nodes interact.
\\\\
The next four sections will dive into great detail about the texhnical aspects that enable the moddeling approach.
\\
First, section \ref{sec:Method:LSM:EuclideanLatentSpace} covers the actual two-dimensional latent space used in this project, namely Euclidean latent space. 
squared Euclidean distance is used as the distance measure for the intensity function, which governs the intensity of interaction, and so this will be covered in detail in section \ref{sec:Method:LSM:SquaredEuclideanDistance}.
\\
Second, a deep dive into Poisson statistics will be presented in section \ref{sec:Method:Poisson}.
The intensity of interaction between any pair of two nodes in a given TDGN is associated with it's own Poisson point process, and therefor essential to understand.
%Synthetic data generation also relies on Poisson statistics, and so this will also be explained fully.
\\
Thirdly, in section \ref{sec:Method:IntensityFunc}, the intensity function is explained in detail.
This function relies on the squared Euclidean distance, and introduces the Bias term, and is what converts between intensity of interaction and pairwise reciprocal distance of nodes in Euclidean latent space.
\\
Fourth, the likelihood function that is used for computing the log-likelihood on which the model optimizes and evaluates is explained in section \ref{sec:Method:LikelihoodFunc}.
\\
All of this is put together, and a thorough technical explanation of the Piecewise Constant Velocity Model is given.
