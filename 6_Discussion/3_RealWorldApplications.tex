\subsection{Real World Applications}
\label{sec:Discussion:UseCases}
The proposed SCVM represents the possibility of modelling the interactions in larger dynamical networks in a two-dimensional Euclidean latent space using velocity dynamics. 
While not many dynamic networks are based in these dynamics, the SCVM enables for a relatively interpretable and explainable modelling of the dynamic network, based on node interaction data, and can potentially give much insight as to how node relationships change over time.
The SCVM provides an opportunity for visually understanding complicated dynamic networks, and also to use the latent space representation as a form of embedding on which machine learning models can be trained.
To exemplify this, a couple of interesting real life use cases are briefly presented below.

\subsubsection{Logistics - Hospitals}
\label{sec:Discussion:UseCases:Hospitals}
Optimization of logistics is a very real challenge for organizations with large complex work environments such as hospitals. 
A lot of logging are typically done in these environments such as treatment times of patients or hospital porter movements. 
The SCVM could be utilized on such data in order to possibly provide insights into the dynamics between individual hospital workers and the patients.
As such, the time nurses for instance allocate to different patients and tasks could more closely be inspected and understood.
The embedding that latent space represents could potentially also be used to anticipate which interactions are most likely to occur and plan in advance to smoothen work flow.


\subsubsection{Future Contacts}
\label{sec:Discussion:UseCases:Marketing}
The SCVM could also be applied to social network data to try and determine likely future contacts for people in the network. 
This might be possible by looking at nodes that are close to each other in the latent space but do not yet interact. 
If these nodes are near each other but do not directly interact, it means they both have some other nodes in common which they interact with, perhaps a common friend. 
In this case they could be likely to meet at a future event and thereby interact.

\subsection{Node Classification on Latent Space}
A third option could be to train machine learning models on the SCVM latent space representation for instance by labeling some nodes and training a supervised model to classify nodes based on their positions in the latent space. 
The latent space representation could also be used in an unsupervised learning setting to help cluster and then label similar nodes based on their interactions.