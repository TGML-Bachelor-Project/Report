

\subsubsection{Introduction}
\label{sec:Method:VModel:IntensityFuncIntro}

The quintessential piece of math underlying the velocity modelling approach of this project is the intensity function, which converts between intensity of interaction between a given pair of nodes in a TDGN and distance between these nodes in latent space.
\\\\
The next three sections will dive into great detail about the aspects that make up this intensity function.
\\
First, the actual two-dimensional latent space used in this project, namely Euclidean latent space, will be covered in section \ref{sec:Method:LSM:EuclideanLatentSpace}, in order to understand how and why it enables the modelling approach of this project. 
As Squared Euclidean distance is used as the measure governing the intensity of interaction, this will be covered in detail in section \ref{sec:Method:LSM:SquaredEuclideanDistance}.
\\
Second, a deep dive into Poisson statistics will be presented in section \ref{sec:Method:Poisson}.
The intensity of interaction between any pair of two nodes in a given TDGN is associated with it's own Poisson point process, and therefor an essential part of the intensity function.
%Synthetic data generation also relies on Poisson statistics, and so this will also be explained fully.
\\
Thirdly, the intensity function which is used in order to compute the likelihood function is explained in detail.
This function relies on the Squared Euclidean distance, and introduces the Bias term, in order to ground intensity of interaction between nodes in the Euclidean latent space.
\\
Fourth, the likelihood function that is used for computing the log-likelihood on which the model optimizes and evaluates is derived and explained.
\\
Lastly, this is all put together and a thorough, much deeper technical explanation of the constant velocity model is given.