\subsection{Graphs and Networks}
\label{sec:Method:Graphs}
The most common way of describing a network of anything, that could be of people on Facebook, the electrical grid, distribution of goods from warehouses to stares etc. is to use a graph.
A graph is a representation that consists of a set of nodes, the entities whose interaction we are trying to depict, and edges which show these interactions.





\subsubsection{Static Graphs}
\label{sec:Method:Graphs:StaticGraphs}
The most common type of graph is, as mentioned, the static graph, that does not change whatsoever.


\subsubsection{Dynamic Graphs}
\label{sec:Method:Graphs:DynamicGraphs}
Not all graphs are static though, and in many cases, having graphs be non-static means that they make for a much better representation of the network they are mapping.
\\
The dynamics differ between graphs, and though this project focuses entirely on graphs that are dynamic in the temporal sense, i.e. they change over time, this is not always the case.



\subsubsection{Temporally Dynamic Graphs}
\label{sec:Method:Graphs:TemporallyDynamicGraphs}
In this project, the temporally dynamic graph network (TDGN), serves as the foundation on which modelling will occur.
The key distinction from static graphs is, that these networks are dynamic as they change through time.










