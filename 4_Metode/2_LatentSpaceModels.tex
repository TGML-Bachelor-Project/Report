\subsection{Latent Space Models}
\label{sec:Method:LSM}
This subsection delves with explaining and understanding the latent space modelling approach.

A latent space, also refereed to as a latent embedding space, serves as an embedding in which a set of entities, in this project the nodes of a dynamic network, who resemble each other more closely are positioned more closer in the latent space.
For a given set of eight entities, these could be human beings with a set of features entailing age, height, weight, eye and hair color etc., their mutual resemblance could be represented in two-dimensional latent space by their reciprocal distances.  

\begin{figure}[H]
    \centering
    \includegraphics[width=0.8\textwidth]{0_images/latentSpaceIllustration.png}
    \caption{Eight humans placed on two-dimensional latent space, their reciprocal distances respresenting their resemblance.}
    \label{fig:RLdataset3}
\end{figure}
For this project, this resemblance is based on the intensity of interaction, defined by the intensity function which is described later in section \ref{sec:Method:IntensityFunc}.
The latent space, being a two-dimensional space, serves as a dimensionality reduction, given the fact that the interaction data is of much higher dimensions.

Latent space models, or latent distance models as they are also referred to, have proved good in modelling higher dimensional data in lower dimensions, in this project two dimensions \cite{Gourieroux2021ScalableNetworks}.
As mentioned in section \ref{sec:Intro:RelatedWork} about related work, the latent space approach by Sarkar and Moore et. al. modelled using positions in latent space \cite{Sarkar2005DynamicModels}.
Tomerup et. al. \cite{Tommerup2021LearningNetworks} expanded their modelling approach by attributing the nodes of a dynamic network with simple Newtonian dynamics of motion, even though most dynamic networks will not be governed by Newtonian dynamics.
This project expands the capabilites of the Newtonian dynamics approach in order to better the modelling ability of these networks which are not Newtonian in nature.


\subsubsection{Euclidean Latent Space}
\label{sec:Method:LSM:EuclideanLatentSpace}
The Euclidean latent space is latent space governed by Euclidean geometry in two dimensions.
This means any measure of position, distance and velocity are understood as Euclidean and can be computed as such. 
Taking an offset in the nodes of a TDGN, and the modelling approach of this project, the properties of Euclidean space will be explained below.
\\\\
Given a TDGN consisting of $N$ nodes, these are all placed in the Euclidean latent space. 
Positions in the Euclidean latent space are denoted as two-dimensional coordinates, and hence each node in the given TDGN is assigned a position, expressed for node $u$ as:
$z_u = \begin{pmatrix}
x_u, y_u
\end{pmatrix}^T$.
\\
As the project deals with temporally dynamic graph networks, the positions of nodes are temporally dependant. 
In order to accommodate for this, each node is assigned one or more velocity vectors, which entails that they move along the trajectory of a constant velocity over a given timespan.
For node $u$, the velocity vector is expressed as:
$v_u = \begin{pmatrix}
v_{x,u}, v_{y,u}
\end{pmatrix}^T$.
\\
By having a starting position, $z_u$, as well as a constant velocity, $v_u$, the position of a node at any time, $t$, is given by: 

\begin{equation}
    \textbf{z}_u(t) = \begin{pmatrix}
    x_u\\
    y_u
    \end{pmatrix}
    +
    \begin{pmatrix}
    v_{x,u}\\
    v_{y,u}
    \end{pmatrix}
    t
    = 
    \begin{pmatrix}
    x_u + v_{x,u}t\\
    y_u + v_{y,u}t
    \end{pmatrix}
\end{equation}
In the Euclidean latent space, based on the given positions, it is possible to compute distances using basic Pythagorean mathematics. 
In order to find the distance between two nodes, the most straightforward approach is to compute their reciprocal Euclidean distance.
The Euclidean distance at the starting position, ie. disregarding time, between node $u$ and $v$, can be computed from the following expression:

\begin{equation}
    ||z_u - z_v||_2
    = 
    \sqrt{(x_u - x_v)^2 + (y_u - y_v)^2}
\end{equation}
As seen above though, the positions of nodes \textbf{are} time dependant, and hence this carries over to the distance measure, which is expressed as:

\begin{equation}
    ||\textbf{z}_u(t) - \textbf{z}_v(t)||_2
    = 
    \sqrt{((x_u + v_{x,u}t) - (x_v + v_{x,v}t))^2 + ((y_u + v_{y,u}t) - (y_v + v_{y,v}t))^2}
\end{equation}
For mathematical/computational reasons, which are explained in detail under section \ref{sec:Method:IntensityFunc:IntegralIntensityFunc}, this project utilizes the squared Euclidean distance as distance measure.
This is written is a simplified form below:

\begin{equation} 
||\textbf{z}_u(t) - \textbf{z}_v(t)||_2^2
= 
(x_u - x_v + (v_{x,u} - v_{x,v})t)^2 + (y_u - y_v + ( v_{y,u} - v_{y,v})t)^2
\label{eq:SquaredEuclideanDistance}
\end{equation}



