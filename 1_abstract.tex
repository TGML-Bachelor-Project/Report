\subsection*{Abstract}

This project proposes the Stepwise Constant Velocity Model (SCVM) which models dynamic networks in two-dimensional Euclidean latent space by binning the time dimension into steps and then fitting constant velocities for the nodes in each step.
The SCVM is implemented in Pytorch with a vectorized training setup in order to utilize the power of parallelization on a CUDA system.
\\\\
The proposed model is evaluated in terms of how well it fits synthesized dynamic networks, where the true parameters of the networks are known and as to have ground truths for comparison purposes. 
Also, the model is evaluated on real data using accuracy scores for node interaction prediction. 
While the predictive performance shows underwhelming results that need further investigation, the SCVM model shows promising results in regards to fitted network interaction intensities.
\\\\
Running times are evaluated for the SCVM running on CPU and CUDA, the results concluding that the implemented vectorized training setup greatly improves running times when utilizing CUDA.
This scalable implementation enables for the modelling of larger dynamic networks than would have been feasible with a non-vectorized training setup.
\\\\
Real dynamic networks are visualized through the creation of animations, which depict the SCVM-modelled stepwise velocity dynamics.
These animations are improved through the implementation of non-disruptive corrections of node positions, drift, and rotation, elevating the interpretability and explainability of the visualized dynamic network.
\\\\
The project discusses the results regarding the posed research questions, and argues for the investigation of a number of additions to the SCVM which could improve its modelling capabilities, scalability and explainability.
Lastly, a brief discussion of some potential real life use cases of the SCVM is given.
